\PassOptionsToPackage{draft}{hyperref}
\documentclass{article}


\usepackage{arxiv}

%%%%%%%%%%%%%%%%%%%%%%%%% 
%Packages
%%%%%%%%%%%%%%%%%%%%%%%%%
\usepackage[utf8]{inputenc} % allow utf-8 input
\usepackage[T1]{fontenc}    % use 8-bit T1 fonts
\usepackage{hyperref}       % hyperlinks
\usepackage{url}            % simple URL typesetting
\usepackage{booktabs}       % professional-quality tables
\usepackage{amsfonts}       % blackboard math symbols
\usepackage{nicefrac}       % compact symbols for 1/2, etc.
\usepackage{microtype}      % microtypography
\usepackage{lipsum}
\usepackage[dvips]{graphicx}
\usepackage{caption}
\usepackage{amssymb}
\usepackage{amsmath}
%\usepackage{subfigure}
%\usepackage{cite}
\usepackage{stfloats} 
\usepackage{color}
\graphicspath{{figures/}}
\usepackage{subcaption}
\usepackage{bm}
\usepackage{framed} % or, "mdframed"
\usepackage[framed]{ntheorem}
\usepackage{algorithm,algorithmic}
\usepackage{multirow}


%%%%%%%%%%%%%%%%%%%%%%%%% 
%Commands
%%%%%%%%%%%%%%%%%%%%%%%%%
\newcommand{\fig}[1]{
	\centerline{\includegraphics[width=9cm]{#1.eps}}
}
\newcommand{\twofig}[2]{
	\centerline{\includegraphics[width=7cm]{#1.eps}
		\includegraphics[width=7
		cm]{#2.eps}}
}
\newcommand{\E}{\mathrm{E}}
\newcommand{\Var}{\mathrm{Var}}
\newcommand{\Cov}{\mathrm{Cov}}
\DeclareMathOperator*{\argmin}{arg\,min} 
\DeclareMathOperator*{\argmax}{arg\,max} 
\graphicspath{{./fig/}}
\newframedtheorem{frm-thm}{Theorem}

\title{Exploring positive noise in estimation theory}


\author{
%  David S.~Hippocampus\thanks{Use footnote for providing further
%    information about author (webpage, alternative
%    address)---\emph{not} for acknowledging funding agencies.} \\
%  Department of Computer Science\\
%  Cranberry-Lemon University\\
%  Pittsburgh, PA 15213 \\
%  \texttt{hippo@cs.cranberry-lemon.edu} \\
%  %% examples of more authors
%   \And
% Elias D.~Striatum \\
%  Department of Electrical Engineering\\
%  Mount-Sheikh University\\
%  Santa Narimana, Levand \\
%  \texttt{stariate@ee.mount-sheikh.edu} \\
  %% \AND
  %% Coauthor \\
  %% Affiliation \\
  %% Address \\
  %% \texttt{email} \\
  %% \And
  %% Coauthor \\
  %% Affiliation \\
  %% Address \\
  %% \texttt{email} \\
  %% \And
  %% Coauthor \\
  %% Affiliation \\
  %% Address \\
  %% \texttt{email} \\
}

\begin{document}
For the estimation problem $y_k=x+e_k$, where $e_k\sim\mathcal{U}[0,b]$, we want to find the MVU estimator. We first represent the PDF of the data as
%
%
\begin{subequations}
\begin{align}
p(y_k;x,b) = \left\{\begin{matrix}
\frac{1}{b} & x<y_k<x+b \\ 
0 & \mathrm{otherwise} 
\end{matrix}\right.	
\end{align}
%
%
or alternatively
\begin{align}
p(y_k;x,b) = \frac{1}{b}\left[\sigma(y_k-x) - \sigma(y_k-(x+b))\right]
\end{align}
%
%
Hence,
	%
%
\begin{align}
p(\bm{y};x,b) &= \frac{1}{b^N}\prod_{k=1}^{N}\left[\sigma(y_k-x) - \sigma(y_k-x-b)\right]\nonumber\\
&=\frac{1}{b^N}\left[\sigma(\min y_k -x) - \sigma(\max y_k -x -b)\right]%\nonumber\\
%&=g\left(T_1(\bm{y}),T_2(\bm{y}),x\right)\cdot h(\bm{y}),
\label{eq:unknown}
\end{align}
%
%
%that can also be written, alternatively, in terms of the product of the two step functions
%%
%%
%\begin{align}
%p(\bm{y};x,b) = \frac{1}{b^N}\sigma(x+b-\max y_k)\sigma(\min y_k -x)
%\end{align}
%
%	
\end{subequations}	

Since we cannot approach the problem of finding MVU using CRLB, I use the Rao-Blackwell-Lehmann-Scheffe theorem. The theorem gives that for any unbiased estimator $\tilde{\bm{\theta}}$ and sufficient statistics $\bm{T}(\bm{y})$, $\hat{\bm{\theta}}=\E(\tilde{\bm{\theta}}\mid\bm{T}(\bm{y}))$ is unbiased and $\Var(\hat{\bm{\theta}})\leq\Var(\tilde{\bm{\theta}})$. Additionally, if $\bm{T}(\bm{y})$ is complete, then $\hat{\bm{\theta}}$ is MVU.

It is also shown in~\cite{book:ET_kay_93} that if the dimension of the sufficient statistics is equal to the dimension of the parameter, then the MVU estimator is given by $\hat{\bm{\theta}}=\bm{g}(\bm{T}(\bm{y}))$ for any function $\bm{g}(\cdot)$ that satisfy
%
%
\begin{align}
\E(\bm{g}(\bm{T})) = \bm{\theta}
\end{align}
%
%
Hence, if we can find the sufficient statistics, and if we can show that it is complete, then we can find the MVU estimator. For that, we can use the Neyman-Fisher theorem that gives the sufficient statistic $\bm{T}(\bm{y})$ if we can factorize the PDF as 
%
%
\begin{align}
p(\bm{y};\theta) = g(T(\bm{y}),\theta)h(\bm{y})
\end{align}
%
%
In our problem, from~\eqref{eq:unknown} for $\bm{\theta}=[x,b]^\top$ we can say $h(\bm{y})=1$ and 
%
%
\begin{subequations}\label{eq:sufficient_statistic}
	%
	%
	\begin{align}
	\bm{T}(\bm{y}) = \begin{bmatrix}
	\min y \\ \max y
	\end{bmatrix} = \begin{bmatrix}
	T_1(\bm{y}) \\ T_2(\bm{y})
	\end{bmatrix}
	\end{align}
	%
	%
	and
	%
	%
	\begin{align}
	\E(\bm{T}(\bm{y})) = \begin{bmatrix}
	x+\frac{b}{N+1}\\\\x+\frac{Nb}{N+1}
	\end{bmatrix}
	\end{align}
	%
	%
\end{subequations}
%
%
Then, any function $\bm{g}(\cdot)$ that makes~\eqref{eq:sufficient_statistic} unbiased would be MVU. 
\section{Known b}
At this point, I am not sure if the same line of reasoning holds. Since now, $b$ is not a parameter anymore. However, if it holds, since we know $b$, we can say that both
%
%
\begin{align}
\hat{x}_{min} = \min y - \frac{b}{N+1}\\
\hat{x}_{max} = \max y - \frac{Nb}{N+1}
\end{align}
%
%
are MVU estimators. In other words, the MVU estimator is not unique.
\section{Unknown b}
This case fits better to the theorem. Since we are only interested in $x$, I try to make a function that makes $\bm{T_1}$ unbiased. Let
%
%
\begin{subequations}
	%
	%
	\begin{align}
	\bm{g}(\bm{T}(\bm{y})) =  \begin{bmatrix}
	T_1-\frac{1}{N}T_2\\\\T_2
	\end{bmatrix}
	\end{align}
	%
	%
	this gives
	%
	%
	\begin{align}
	\E(\bm{g}(\bm{T}(\bm{y}))) = \begin{bmatrix}
	\frac{N-1}{N}x\\\\\times
	\end{bmatrix}
	\end{align}
	%
	%
	Hence, an unbiased estimate of $x$ would be given by multiplying first row by $\frac{N}{N-1}$
	%
	%
	\begin{align}
	\hat{x}_{MVU} &= \frac{N}{N-1}\left(T_1-\frac{1}{N}T_2\right) = \frac{N}{N-1}T_1 - \frac{1}{N-1}T_2\nonumber\\
	&= \frac{N}{N-1}\min y_k - \frac{1}{N-1} \max y_k
	\end{align}
	%
	%
\end{subequations}


\bibliographystyle{unsrt}
%\bibliography{references}
%\bibliographystyle{IEEEtran}
\bibliography{./ref/arxiv_18}
\end{document}
